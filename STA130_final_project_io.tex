\documentclass[ignorenonframetext,]{beamer}
\setbeamertemplate{caption}[numbered]
\setbeamertemplate{caption label separator}{: }
\setbeamercolor{caption name}{fg=normal text.fg}
\beamertemplatenavigationsymbolsempty
\usepackage{lmodern}
\usepackage{amssymb,amsmath}
\usepackage{ifxetex,ifluatex}
\usepackage{fixltx2e} % provides \textsubscript
\ifnum 0\ifxetex 1\fi\ifluatex 1\fi=0 % if pdftex
  \usepackage[T1]{fontenc}
  \usepackage[utf8]{inputenc}
\else % if luatex or xelatex
  \ifxetex
    \usepackage{mathspec}
  \else
    \usepackage{fontspec}
  \fi
  \defaultfontfeatures{Ligatures=TeX,Scale=MatchLowercase}
\fi
% use upquote if available, for straight quotes in verbatim environments
\IfFileExists{upquote.sty}{\usepackage{upquote}}{}
% use microtype if available
\IfFileExists{microtype.sty}{%
\usepackage{microtype}
\UseMicrotypeSet[protrusion]{basicmath} % disable protrusion for tt fonts
}{}
\newif\ifbibliography
\hypersetup{
            pdftitle={Most Harzardous Driving Area of Canada?},
            pdfauthor={CEDLucien},
            pdfborder={0 0 0},
            breaklinks=true}
\urlstyle{same}  % don't use monospace font for urls
\usepackage{graphicx,grffile}
\makeatletter
\def\maxwidth{\ifdim\Gin@nat@width>\linewidth\linewidth\else\Gin@nat@width\fi}
\def\maxheight{\ifdim\Gin@nat@height>\textheight0.8\textheight\else\Gin@nat@height\fi}
\makeatother
% Scale images if necessary, so that they will not overflow the page
% margins by default, and it is still possible to overwrite the defaults
% using explicit options in \includegraphics[width, height, ...]{}
\setkeys{Gin}{width=\maxwidth,height=\maxheight,keepaspectratio}

% Prevent slide breaks in the middle of a paragraph:
\widowpenalties 1 10000
\raggedbottom

\AtBeginPart{
  \let\insertpartnumber\relax
  \let\partname\relax
  \frame{\partpage}
}
\AtBeginSection{
  \ifbibliography
  \else
    \let\insertsectionnumber\relax
    \let\sectionname\relax
    \frame{\sectionpage}
  \fi
}
\AtBeginSubsection{
  \let\insertsubsectionnumber\relax
  \let\subsectionname\relax
  \frame{\subsectionpage}
}

\setlength{\parindent}{0pt}
\setlength{\parskip}{6pt plus 2pt minus 1pt}
\setlength{\emergencystretch}{3em}  % prevent overfull lines
\providecommand{\tightlist}{%
  \setlength{\itemsep}{0pt}\setlength{\parskip}{0pt}}
\setcounter{secnumdepth}{0}

\title{Most Harzardous Driving Area of Canada?}
\subtitle{Redacted}
\author{CEDLucien}
\date{}

\begin{document}
\frame{\titlepage}

\begin{frame}{Introduction}

Hundreds of thousands of Canadians get injured each year in car
accidents, and thousands die each year from them. With this many
accidents, it would be extremely beneficial to identify dangerous
driving areas. Using the dataset ``Hazardous Driving Areas'' from
GeoTab, we will investigate determine the most dangerous driving area in
Canada, and more.

Geotab's ``Hazardous Driving Areas'' dataset publishes real-time and
historical incident that captures both accident and near-miss events,
for example, sudden braking. It provides measurements related to driving
incidents, and generates a severity score to rank hazardous areas around
the world.

\end{frame}

\begin{frame}{Objectives}

\begin{itemize}
\tightlist
\item
  The purpose of this analysis is to determine the most hazardous
  driving areas in Canada.
\item
  We define an area as a hazardous driving area if its severity score is
  larger than the average severity score and the number of incidents is
  higher than the average number of incidents.
\end{itemize}

\end{frame}

\begin{frame}{Data Summary}

\begin{itemize}
\tightlist
\item
  \textbf{New Variables}
\item
  Proportion of type of incidents to total incidents
\item
  Total incidents of each type in each province
\item
  Is an area hazardous
\item
  \textbf{Modifications}
\item
  Joined ``Hazardous Driving Areas'' dataset with a ``Population of
  Canada'' dataset
\end{itemize}

\end{frame}

\begin{frame}[fragile]{Population}

\begin{verbatim}
## # A tibble: 10 x 2
##    State                     Population
##    <chr>                          <dbl>
##  1 Canada                      36963854
##  2 Newfoundland and Labrador     527613
##  3 Nova Scotia                   957470
##  4 New Brunswick                 760744
##  5 Quebec                       8439925
##  6 Ontario                     14318750
##  7 Manitoba                     1346993
##  8 Saskatchewan                 1169752
##  9 Alberta                      4318772
## 10 British Columbia             4849442
\end{verbatim}

\end{frame}

\begin{frame}[fragile]{Total Areas in Each Province}

\begin{verbatim}
## # A tibble: 10 x 2
##    State                     total
##    <fct>                     <int>
##  1 Alberta                     477
##  2 British Columbia            666
##  3 Manitoba                    601
##  4 New Brunswick               120
##  5 Newfoundland and Labrador    49
##  6 Nova Scotia                 243
##  7 Ontario                    5909
##  8 Prince Edward Island          2
##  9 Quebec                     2128
## 10 Saskatchewan                 46
\end{verbatim}

\end{frame}

\begin{frame}[fragile]{Total Number of Incidents}

\begin{verbatim}
## # A tibble: 9 x 2
##   State                     total_inci
##   <fct>                          <int>
## 1 Alberta                         4968
## 2 British Columbia                6158
## 3 Manitoba                       15412
## 4 New Brunswick                   4467
## 5 Newfoundland and Labrador        977
## 6 Nova Scotia                     6617
## 7 Ontario                       109283
## 8 Quebec                         84791
## 9 Saskatchewan                     516
\end{verbatim}

\end{frame}

\begin{frame}[fragile]{Avgerage Severity Score and Average Number of
Incidents}

\begin{verbatim}
## # A tibble: 10 x 3
##    State                     avg_severity_score avg_num_inci
##    <fct>                                  <dbl>        <dbl>
##  1 Alberta                               0.109         10.4 
##  2 British Columbia                      0.0660         9.25
##  3 Manitoba                              0.0863        25.6 
##  4 New Brunswick                         0.130         37.2 
##  5 Newfoundland and Labrador             0.272         19.9 
##  6 Nova Scotia                           0.0603        27.2 
##  7 Ontario                               0.0808        18.5 
##  8 Quebec                                0.115         39.8 
##  9 Saskatchewan                          0.439         11.2 
## 10 Canada                                0.0912        22.8
\end{verbatim}

\end{frame}

\begin{frame}[fragile]{Proportion of Hazardous Area in Each Province}

\begin{verbatim}
## # A tibble: 9 x 4
##   State                     num_yes total proportion
##   <fct>                       <int> <int>      <dbl>
## 1 Newfoundland and Labrador      28    49       57.1
## 2 Quebec                        882  2128       41.4
## 3 Saskatchewan                   19    46       41.3
## 4 New Brunswick                  46   120       38.3
## 5 Nova Scotia                    62   243       25.5
## 6 Manitoba                      153   601       25.5
## 7 Ontario                      1467  5909       24.8
## 8 Alberta                        96   477       20.1
## 9 British Columbia              130   666       19.5
\end{verbatim}

\end{frame}

\begin{frame}{Statistical Methods}

\begin{itemize}
\tightlist
\item
  Binary: Using ifelse to mutate a new binary variable based on our
  definition for hazadous area.
\item
  ggplot: Scatterplot, Barplot
\item
  Classification Tree: Using HdtIncident and MdtIncidentt as variables
  and sets three level of HdtIncident (None, Some and Many) to predict
  the H\_driving by classification tree.
\item
  ROC: Using the ROCR to calculate each threshold value for our
  classification tree
\end{itemize}

\end{frame}

\begin{frame}{Average Sererity Score and Average Number of Incidents
Scatterpolt}

\includegraphics{STA130_final_project_io_files/figure-beamer/unnamed-chunk-7-1.pdf}

\end{frame}

\begin{frame}

\includegraphics{STA130_final_project_io_files/figure-beamer/unnamed-chunk-8-1.pdf}

\end{frame}

\begin{frame}

\includegraphics{STA130_final_project_io_files/figure-beamer/unnamed-chunk-9-1.pdf}

\end{frame}

\begin{frame}

\includegraphics{STA130_final_project_io_files/figure-beamer/unnamed-chunk-10-1.pdf}

\end{frame}

\begin{frame}

\includegraphics{STA130_final_project_io_files/figure-beamer/unnamed-chunk-11-1.pdf}

\end{frame}

\begin{frame}

\includegraphics{STA130_final_project_io_files/figure-beamer/unnamed-chunk-12-1.pdf}

\end{frame}

\begin{frame}{Classification Tree}

\includegraphics{STA130_final_project_io_files/figure-beamer/unnamed-chunk-13-1.pdf}

\end{frame}

\begin{frame}{ROC}

\includegraphics{STA130_final_project_io_files/figure-beamer/unnamed-chunk-14-1.pdf}

\end{frame}

\begin{frame}{Results}

\begin{itemize}
\tightlist
\item
  Base on the data, we discovered that the Saskatchewan and the
  Newfoundland and Labrador are the most two hazardous driving area
\item
  However, since they have the smallest dataset which their datasets are
  less than 50, so we cannot use them to satisfy our prediction.
\item
  As the results, we find Quebec is the most hazardous driving area.
\item
  Classification Tree
\item
  Two significant main effects -- the number of incidents involving a
  medium-duty truck and a heavy-duty truck in Quebec state (from the
  figure P\_Hdt and P\_Mdt).
\end{itemize}

\end{frame}

\begin{frame}[fragile]{Proportion of Hazardous Area in Each Province}

\begin{verbatim}
## # A tibble: 9 x 4
##   State                     num_yes total proportion
##   <fct>                       <int> <int>      <dbl>
## 1 Newfoundland and Labrador      28    49       57.1
## 2 Quebec                        882  2128       41.4
## 3 Saskatchewan                   19    46       41.3
## 4 New Brunswick                  46   120       38.3
## 5 Nova Scotia                    62   243       25.5
## 6 Manitoba                      153   601       25.5
## 7 Ontario                      1467  5909       24.8
## 8 Alberta                        96   477       20.1
## 9 British Columbia              130   666       19.5
\end{verbatim}

\end{frame}

\begin{frame}

\includegraphics{STA130_final_project_io_files/figure-beamer/unnamed-chunk-16-1.pdf}

\end{frame}

\begin{frame}

\includegraphics{STA130_final_project_io_files/figure-beamer/unnamed-chunk-17-1.pdf}

\end{frame}

\begin{frame}{Conclusion}

\begin{itemize}
\tightlist
\item
  \textbf{conclusion}: Quebec is the most Hazardous driving province
\item
  -\textgreater{} Highest probability of Heavy and Medium Duty Truck
  Incidents
\item
  \textbf{limitation} : may only work for this specific data set
\item
  \textbf{error} : Newfoundland and Labrador has the highest proportion
  of hazardous driving area, it does not match our definition of
  hazardous driving area
\end{itemize}

\end{frame}

\begin{frame}{Reference}

``Canadian Motor Vehicle Traffic Collision Statistics: 2015.'' Goverment
of Canada -Transport Canada, 26 May 2017,
www.tc.gc.ca/eng/motorvehiclesafety/tp-tp3322-2015-1487.html.

``Hazardous Driving Areas.'' Geotab Data, 13 Mar. 2018,
data.geotab.com/urban-infrastructure/hazardous-driving.

Statistics Canada. ``Population and Dwelling Count Highlight Tables,
2011 Census.''Government of Canada, Statistics Canada, 9 Aug. 2016,
www12.statcan.ca/census-recensement/2011/dp-pd/hlt-fst/pd-pl/Table-Tableau.cfm?LANG=Eng\&T=301\&SR=1\&S=3\&O=D\&RPP=25\&PR=0\&CMA=0.

\end{frame}

\end{document}
